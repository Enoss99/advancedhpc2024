\documentclass{article}
\usepackage{graphicx}
\usepackage{amsmath}
\usepackage{hyperref}
\usepackage{listings}

\title{Lab Report 6: MAP}
\author{Julien}
\date{\today}

\begin{document}

\maketitle


\section{Implementation}
\textbf{Image Loading:} Images are loaded in RGB and converted to grayscale using:
\[
\text{Grayscale} = 0.2989 R + 0.5870 G + 0.1140 B
\]

\textbf{CUDA Kernels:} 
\begin{itemize}
    \item \texttt{binarize\_image}: Sets each pixel to 0 or 1 based on a threshold.
    \item \texttt{adjust\_brightness}: Adds an offset to each pixel.
    \item \texttt{blend\_images}: Averages two images based on a blend coefficient.
\end{itemize}

\section{Block Size Experiments}
Block sizes $(8 \times 8)$, $(16 \times 16)$, and $(32 \times 32)$ were tested to measure performance:

\begin{table}[h!]
\centering
\begin{tabular}{|c|c|c|c|}
\hline
\textbf{Block Size} & \textbf{Binarization (s)} & \textbf{Brightness (s)} & \textbf{Blending (s)} \\
\hline
$8 \times 8$ & 0.032 & 0.034 & 0.035 \\
$16 \times 16$ & 0.018 & 0.020 & 0.021 \\
$32 \times 32$ & 0.015 & 0.016 & 0.017 \\
\hline
\end{tabular}
\end{table}

\section{Conclusion}
Larger blocks generally improved speed, with $(32 \times 32)$ performing best.

\end{document}
